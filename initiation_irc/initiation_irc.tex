\documentclass{beamer}

\usetheme{Madrid}
\usepackage[utf8]{inputenc}
\usepackage[T1]{fontenc}
\usepackage[francais]{babel}
\usepackage{textpos}
\usepackage{hyperref}

% Get rid of the navigation bars
\beamertemplatenavigationsymbolsempty
% Change itemization markers to 'squares'
\useinnertheme{rectangles}
\setbeamertemplate{itemize items}[triangle]

\author[DePierre]{Tao "DePierre" S.}
\title{Initiation IRC}
\institute[HackGyver]{\includegraphics[height=3cm,width=3cm]{images/LogoHackGyver.png}\\\large{Le hackerspace de Belfort}}
\date{10 avril 2012}
\logo{\includegraphics[height=1.5cm,width=1.5cm]{images/LogoHackGyver.png}}

\begin{document}

% Title page
\begin{frame}[plain]
	\maketitle
\end{frame}

% Table of contents
\begin{frame}{Plan}
	\tableofcontents
\end{frame}

% Options for the display of the table of contents
\AtBeginSubsection[]
{
   \begin{frame}
        \frametitle{Initiation IRC}
        \tableofcontents[ 
    		currentsubsection, 
    		currentsubsections, 
    	] 
   \end{frame}
}

% First part: in few words
\section{En quelques mots}

\begin{frame}{En quelques mots}
\begin{itemize}
	\itemsep1.3em
		\item Création 1988
		\item Protocole ouvert basé sur TCP
		\begin{itemize}
			\item Port 6667 par défaut
		\end{itemize}
		\item Composé de
		\begin{itemize}
			\item Serveurs
			\item Clients (présentés ici)
		\end{itemize}
		\item \href{https://tools.ietf.org/html/rfc1459}{RFC irc2.4.0}
		\begin{itemize}
			\item Non respectée par tous
			\item Plusieurs révisions effectuées
		\end{itemize}
\end{itemize}
\end{frame}

\begin{frame}{En quelques mots}
\begin{itemize}	
	\itemsep1.3em
		\item Communication entre les serveurs
		\begin{itemize}
			\item Étend le réseau
		\end{itemize}
		\item Salons de discutions (a.k.a channel)
		\item Chat privé possible
		\item Échanges de fichiers possibles
		\begin{itemize}
			\item Nécessite ouverture de ports
			\item Non détaillé ici
		\end{itemize}
\end{itemize}
\end{frame}

% Second part: some clients
\section{Clients connus}

	\subsection{XChat} % XChat
\begin{frame}{XChat}
\begin{columns}
	\column{.5\textwidth}
		\begin{itemize}	
			\itemsep1.3em
				\item Client pour
				\begin{itemize}
					\item Linux
					\item Mac OS X
					\item Windows
				\end{itemize}
				\item Interface graphique (GTK+)
				\item Système de scripts/plugins
				\item Code source sous licence GPL
		\end{itemize}
	\column{1\textwidth}
		\href{https://commons.wikimedia.org/wiki/File:Logo_xchat.png}{\hspace{3cm} \includegraphics[width=0.15\textwidth]{images/Logo_xchat.png}}
\end{columns}
\end{frame}

\begin{frame}{XChat}
\begin{itemize}	
	\itemsep1.3em
		\item Avantages
		\begin{itemize}
			\item Très complet
			\item Très user-friendly
			\item Ultra personnalisable
		\end{itemize}
		\item Inconvénients
		\begin{itemize}
			\item Assez lourd (relatif)
			\item Configuration plus difficile à transporter
		\end{itemize}
\end{itemize}
\begin{center}
	\textbf{Accueil} : \url{http://www.silverex.org/}
\end{center}
\end{frame}

	\subsection{AdiIRC} % AdiIRC
\begin{frame}{AdiIRC}
\begin{columns}
	\column{.5\textwidth}
		\begin{itemize}	
			\itemsep1.3em
				\item Client pour
				\begin{itemize}
					\item Windows
				\end{itemize}
				\item Interface graphique (.NET 2.0)
				\item Système de scripts/plugins
		\end{itemize}
	\column{1\textwidth}
		\href{https://danielsan89.deviantart.com/art/adiIRC-dock-icon-69465482}{\hspace{3cm} \includegraphics[width=0.15\textwidth]{images/adiIRC_dock_icon_by_Danielsan89.png}}
\end{columns}
\end{frame}

\begin{frame}{AdiIRC}
\begin{itemize}	
	\itemsep1.3em
		\item Avantages
		\begin{itemize}
			\item Très léger
			\item User-friendly
			\item Configuration très simple à transporter
		\end{itemize}
		\item Inconvénients
		\begin{itemize}
			\item Ne fonctionne que pour Windows
			\item Vitesse de développement variable
			\item Documentation faible pour les plugins
		\end{itemize}
\end{itemize}
\begin{center}
	\textbf{Accueil} : \url{http://www.adiirc.com/}
\end{center}
\end{frame}

	\subsection{WeeChat} % WeeChat
\begin{frame}{WeeChat}
\begin{columns}
	\column{.5\textwidth}
		\begin{itemize}	
			\itemsep1.3em
				\item Client pour
				\begin{itemize}
					\item BSD
					\item Linux
					\item Mac OS X
				\end{itemize}
				\item Pas d'interface graphique
				\item Système de scripts/plugins
				\item Logiciel libre
		\end{itemize}
	\column{1\textwidth}
		\href{https://commons.wikimedia.org/wiki/File:Weechat_logo.png}{\hspace{3cm} \includegraphics[width=0.25\textwidth]{images/Weechat_logo.png}}
\end{columns}
\end{frame}

\begin{frame}{WeeChat}
\begin{itemize}	
	\itemsep1.3em
		\item Avantages
		\begin{itemize}
			\item Ultra léger
			\item Ultra personnalisable
			\item Développement très actif
		\end{itemize}
		\item Inconvénients
		\begin{itemize}
			\item Moins user-friendly
			\item Configuration parfois difficile à transporter
		\end{itemize}
\end{itemize}
\begin{center}
	\textbf{Accueil} : \url{http://www.weechat.org/}
\end{center}
\end{frame}

% Third part: Surival guide for beginners
\section{Commandes de survie}

	\subsection{Serveur} % Server
\begin{frame}{Serveur}
\begin{itemize}	
	\itemsep1.3em
		\item /SERVER <server> [port]
		\begin{itemize}
			\item Se connecter à un serveur IRC
			\item \textit{server} l'adresse du serveur
			\item \textit{port} le port de connexion
			\item \textit{Exemple : /SERVER irc.freenode.org 6667}
		\end{itemize}
		\item /LIST
		\begin{itemize}
			\item Liste les canaux du serveur
		\end{itemize}
\end{itemize}
\end{frame}
	
	\subsection{Soi-même} % Oneself
	
\begin{frame}{Soi-même}
\begin{itemize}	
	\itemsep1.3em
		\item /NICK <nickname>
		\begin{itemize}
			\item Change son pseudo
			\item \textit{nickname} son pseudo
			\item \textit{Exemple : /NICK John}
		\end{itemize}
		\item /ME <stuff\_im\_doing>
		\begin{itemize}
			\item Parle à la troisième personne
			\item \textit{stuff\_im\_doing} ce que vous voulez dire
			\item \textit{Exemple : /ME eat potoates}
		\end{itemize}
		\item /AWAY [reason]
		\begin{itemize}
			\item Se marque absent se marque présent
			\item \textit{reason} la raison de votre absence
			\item \textit{Exemple : /AWAY dodo}
		\end{itemize}
\end{itemize}
\end{frame}

\begin{frame}{Soi-même}
\begin{itemize}	
	\itemsep1.3em
		\item /PART <channel> [, chan2, chan3] [reason]
		\begin{itemize}
			\item Quitte un canal/plusieurs canaux
			\item \textit{channel} le nom du canal
			\item \textit{Exemple : /PART \#hackgyver, \#bouloup}
		\end{itemize}
		\item /QUIT [reason]
		\begin{itemize}
			\item Quitte le serveur
			\item \textit{reason} explique pourquoi
			\item \textit{Exemple : /QUIT go to Paris}
		\end{itemize}
\end{itemize}
\end{frame}
	
	\subsection{Canal} % Channel
\begin{frame}{Canal}
\begin{itemize}	
	\itemsep1.3em
		\item /JOIN <channel>
		\begin{itemize}
			\item Se connecte à un canal
			\item \textit{channel} le nom du canal
			\item \textit{Exemple : /JOIN \#hackgyver}
		\end{itemize}
		\item /MSG <pseudo> <message>
		\begin{itemize}
			\item Envoie un message privé
			\item \textit{pseudo} le pseudo du destinataire
			\item \textit{message} le message à transmettre
			\item \textit{Exemple : /MSG John hello there}
		\end{itemize}
		\item /QUERY <pseudo> [message]
		\begin{itemize}
			\item Ouvre un salon privé
			\item \textit{pseudo} la personne que vous invité dans ce salon
			\item \textit{message} le message à transmettre
			\item \textit{Exemple : /QUERY John}
		\end{itemize}
\end{itemize}
\end{frame}

% Fourth part: Demo time
\section{Demo}
\begin{frame}{Demo}
\begin{center}
	\Huge{DEMO} \\
	\Large{Se connecter à HackGyver sur le serveur freenode}
\end{center}
\end{frame}

% Fifth part: Conclusion
\section{Conclusion}
\begin{frame}{Conclusion}
IRC c'est :
\begin{itemize}
	\itemsep1.3em
		\item Une communauté énorme
		\item De l'aide disponible en masse (\#archlinux, \#debian, \#root-me, ...)
		\item Rapide et simple d'utilisation
		\item De l'information en temps réel
\end{itemize}
\begin{center}
	\Large{Rejoignez notre canal IRC \textbf{HackGyver@irc.freenode.org}}
\end{center}
\end{frame}

% Sixth part: Questions
\section{Questions}
\begin{frame}{Questions ?}
\begin{center}
	\includegraphics[height=3cm,width=3cm]{images/LogoHackGyver.png} \\
	\Huge{Questions ?}
\end{center}
\end{frame}

\end{document}